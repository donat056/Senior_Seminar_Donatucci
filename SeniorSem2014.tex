\documentclass[12pt]{article}

\setlength{\oddsidemargin}{0in}
\setlength{\evensidemargin}{0in}
\setlength{\topmargin}{0in}
\setlength{\headheight}{0in}
\setlength{\headsep}{0in}
\setlength{\textwidth}{6in}
\setlength{\textheight}{9in}
\setlength{\parindent}{0in} 

\usepackage{parskip}
\usepackage{times} %For typeface
\usepackage{graphicx}
\usepackage{algorithm}
\usepackage{algorithm,algorithmic}
\usepackage[justification=centering]{caption}[2007/12/23]
\usepackage{url}
\sloppy

\usepackage{float}
\newfloat{Query}{tbp}{lop}

\newcommand{\inset}[1]{$\in \{ {#1} \}$}

\newcommand{\citep}[1]{\cite{#1}}
\newcommand{\PPLR}[1]{$\eta_M$}
\newcommand{\LLR}[1]{$\eta_L$}

\DeclareGraphicsRule{.tif}{png}{.png}{`convert #1 `dirname #1`/`basename #1 .tif`.png}

\title{Modern Techniques of Energy Optimization for Green Cloud Computing}

\author{
 		David Donatucci\\
        Division of Science and Mathematics\\
        University of Minnesota, Morris\\
        Morris, MN 56267\\
        donat056@morris.umn.edu\\
}
\date{} 

\begin{document}
\pagestyle{plain}

\maketitle

\begin{abstract}

Place abstract here

\end{abstract}

\section{Introduction} \label{sec:intro}

This paper demonstrates the usefulness of graph databases in recording and analyzing data produced by GP systems. A description of genetic programming is provided in Section~\ref{Background}, and Section~\ref{sec:Graph Databases} discusses graph databases. Section~\ref{sec:experiments} provides details on how we set up our experimental runs. The results of our work are presented in Section~\ref{sec:results}, and ideas for future implementation and applications of this work are presented in Section~\ref{sec:conclusion}.

\section{Background} \label{Background}

In order for the reader to fully understand the material presented in this paper, I will present some necessary background information.

\subsection{Cloud Computing}
\label{sec:Cloud Computing}

In the last several years, cloud computing has become a predominate service. Cloud computing pertains to both the applications services provided via the Internet as well as the hardware and systems software in data centers. Inside these data centers are an exuberant amount of servers (the hardware) managed by middleware (software that communicates between servers) which is called the \emph{cloud}.  Usually, large IT firms such as Google or Amazon provide the hardware and system software for these large data centers. However, these large IT firms have extra servers and rent them to smaller companies providing a relatively cheap system to compute in the cloud. 

\subsection{Evolutionary Computation (EC)}
\label{sec:EC}



\section{Graph Databases}
\label{sec:Graph Databases}



\begin{verbatim}
    START parent=node(43)
    MATCH (parent)-[:PARENTOF]->(child)
    RETURN parent, child;
\end{verbatim}


\section{Experimental Setup} 
\label{sec:experiments}

This section explains the details of the configurations used for this research. Subsection~\ref{sec:GPSetup} covers setup of the genetic programming algorithm, and Subsection~\ref{sec:Neo4jSetup} discusses setup of the graph database Neo4j.

\subsection{Genetic Programming Setup}
\label{sec:GPSetup}


\section{Results} \label{sec:results}



\subsection{Number of Initial Individuals With Final Generation Descendants}
\label{sec:numberInitialIndividualsWithDescendants}



\subsection{Effectiveness of Mutation and Crossover}
\label{sec:effectivenessMutationCrossover}



\subsection{Winning Root Ancestry Line Fitness}
\label{sec:WinningRootLineFitness}


\section{Conclusions} \label{sec:conclusion}



\section*{Acknowledgements}

David's work was supported by the Morris Academic Partners program at the University of Minnesota, Morris. Many thanks to Nicholas Cornhill and Emma Ireland for their early help in connecting evolutionary computation systems to Neo4j.

\pagebreak

\bibliographystyle{acm}
\bibliography{annotated_bibliography}

\end{document}